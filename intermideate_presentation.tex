\documentclass[12pt,aspectratio=169,hyperref={pdftex,unicode},xcolor=dvipsnames]{beamer}
\usepackage[english]{babel}
\usepackage[utf8x]{inputenc}
\usepackage[T2A]{fontenc}
\usepackage{cmap}
\usepackage{paratype}
\usepackage{minted}
\usepackage{blkarray} % for examples with code (require parameter -shell-escape)

\usetheme{metropolis}
\usefonttheme[]{professionalfonts}  % prohibit overwriting fonts to beamer
\metroset{numbering=fraction}
\metroset{subsectionpage=progressbar}

\setbeamercolor{frametitle}{fg=black}
\setbeamertemplate{frametitle}
{
    \vspace{3mm}\insertframetitle\par
}
\setbeamertemplate{title separator}{}
\setbeamertemplate{footnote separator}{}

\logo{\vspace{-1.2cm}\includegraphics[width=8mm]{./common/nup-icon.png}\hspace*{1.08\textwidth}}

\institute
{
    \begin{columns}
        \begin{column}{2.5cm}
            \includegraphics[height=30mm,keepaspectraticommon/nup-icon.png]
        \end{column}
        \begin{column}{3cm}
            Neapolis University Paphos
        \end{column}
    \end{columns}
}

\begin{document}

  \begin{frame}[plain]
        \begin{left}

        {\Large\textbf{Content blocking systems in Cyprus}}

            \vspace{5mm}
            \textbf{Dmitrii Stepul}

            {\small Supervisors: I. Agarkov, L. Zacharioudakis}

            \vspace{10mm}
            16.03.2024
        \end{left}

        \vspace{10mm}

        \begin{column}{1cm}
            \includegraphics[height=14mm,keepaspectratio]{./common/nup-logo.png}
        \end{column}
    \end{frame}


    
\begin{frame}
    \frametitle{What is the entire project about?}
    \begin{figure}
    \centering
    \begin{center}
        \includegraphics[height=50mm,keepaspectratio]{images/Cyta_blockpage.png}
    \end{center}
    \caption{Blocking vk.com from the outside when using Cyta sim cards}
    \end{figure}
\end{frame}



\begin{frame}
    \frametitle{What is the entire project about?}
    
          \begin{figure}
    \begin{center}
            \centering\includegraphics[height=50mm,keepaspectratio]{images/rostelecom.png}
    \end{center}
    \caption{Blocking non-Russian IP addresses on the Russian side}
    \label{fig:russian_blockage}
    \end{figure}
\end{frame}

  \begin{frame}
        \frametitle{Novelty}
        There is no ISPs censorship checker for Cyprus.
        \begin{itemize}
            \item Some tools are outdated\footnote{https://github.com/ValdikSS/blockcheck} 
            \item Сurrent research in Cyprus is more related to the legal side of censorship, rather than technical implementation\footnote{Internet Censorship Capabilities in Cyprus: An Investigation of Online Gambling Blocklisting}
            \item Popular open-source tools aren't working in Cyprus \footnote{https://github.com/ValdikSS/GoodbyeDPI}
        \end{itemize}


    \end{frame}


    \begin{frame}{Why is it popular}
        \begin{figure}
            \centering
            \includegraphics[width=0.5\linewidth]{images/ratings.png}
            \caption{Number of entries per blocklist and blocklist type: (C) copyright, (G) gambling, (M) health/medical,
(IP) IP address based, (P) phishing, (T) tobacco. }
            \label{fig:rating}
        \end{figure}
    \end{frame}



    \begin{frame}
        \frametitle{The Problem Statement}
         \begin{enumerate}
            \item To investigate the current methods of blocking content in Internet in Cyprus
            \item  To compare results of blocking with differnet  Cypriot ISPs (like Primetel, Cyta,etc) \footnote{It will be an independent research without cooperation with ISPs}
            \item Code the checker for ISPs and other people
        \end{enumerate}



     

    \end{frame}

    
      \begin{frame}{Deep Packet Inspection}
       State: We are sending GET request to vk.com\footnote{Russian social network which is blocked in EU}  through the Cyta network \\ 
      There are two types of Deep Packet Inspection (DPI): 
        \begin{itemize}
            \item Lazy DPI
            \item Smart DPI
        \end{itemize}
        
            
        \end{frame}



    
\begin{frame}[fragile] % необходимо использовать опцию fragile, чтобы TeX не обрабатывал специальные символы
    \frametitle{Trying to bypass DPI using RFC standards}
    \begin{verbatim}
GET / HTTP/1.0
Host: vk.com
Accept-Encoding: gzip, deflate, br
Connection: close
    \end{verbatim}
     \begin{quote}
        3.2.  Header Fields

        Each header field consists of a case-insensitive field name followed
        by a colon (":"), optional leading whitespace, the field value, and
        optional trailing whitespace.
    \end{quote}
    \vfill
    \tiny{Source: RFC 7230 \textit{HTTP/1.0 Message Syntax and Routing}, June 2014}
\end{frame}
    

    
\begin{frame}[fragile] % необходимо использовать опцию fragile, чтобы TeX не обрабатывал специальные символы
    \frametitle{Trying to bypass DPI using RFC standards}
    \begin{verbatim}
GET / HTTP/1.0
Host: vk.com
Accept-Encoding: gzip, deflate, br
Connection: close
    \end{verbatim}
    \begin{itemize}
        \item Host -> hOst or hOSt
        \item Add spaces and tabulation
        \item Split one packet into two and send it fragmented
        \item Add paddings
    \end{itemize}
\end{frame}



\begin{frame}[fragile]
    \frametitle{Trying to bypass DPI using RFC standards}

    
    \begin{center}

        \begin{itemize}
        \item \textbf{This site can’t be reached due tο compliance with the Council Regulation (EU) 350/2022 and with EU and National Laws, only for as long as necess}\\ 
    \item  \textbf{Your browser sent a request that this server could not understand.}
       \end{itemize}
    \end{center}
\end{frame}


\begin{frame}{SNI-blocking}
    \begin{figure}
        \centering
        \includegraphics[width=0.5\linewidth]{images/sni_blocking_example.png}
        \caption{TLS scheme \footnote{Service-Level Monitoring of HTTPS Traffic, DOI: 10.13140/RG.2.2.32296.67849}}
        \label{fig:enter-label}
    \end{figure}
\end{frame}


\begin{frame}[fragile]
    \frametitle{IP-blocking on Russian websites}
\begin{figure}
    \centering
    \includegraphics[width=0.75\linewidth]{images/у русских блокировки.png}
    \caption{Traceroute to vk.com via Cyta}
    \label{fig:enter-label}
\end{figure}
    \end{frame}



    \begin{frame}[fragile]
    \frametitle{IP-blocking on Russian websites}
    \begin{figure}
        \centering
        \includegraphics[width=0.75\linewidth]{images/traceroute_to_vk_via_CYTA.png}
        \caption{Traceroute to rt.ru via Cyta}
        \label{fig:enter-label}
    \end{figure}
    \end{frame}


    

        
    
  \begin{frame}
        \frametitle{Plans}

        \begin{enumerate}
            \item Nicely organize the results into tabular\graphs 
            \item Try to bypass TLS interception\ SNI-blocking
        \end{enumerate}

        \vspace{5mm}\hrule\vspace{5mm}

        \begin{center}
         Dmitrii Stepul, +35797804220, \\ d.stepul@nup.ac.cy
        \end{center}

    \end{frame}
  

\end{document}